\documentclass[a4paper]{article}
% Kodowanie latain 2
%\usepackage[latin2]{inputenc}
\usepackage[T1]{fontenc}
% Można też użyć UTF-8
\usepackage[utf8]{inputenc}
\usepackage{listings}
% Język
\usepackage[polish]{babel}
% \usepackage[english]{babel}

% Rózne przydatne paczki:
% - znaczki matematyczne
\usepackage{amsmath, amsfonts}
% - wcięcie na początku pierwszego akapitu
\usepackage{indentfirst}
% - komenda \url w wersji nie tworzącej dodatkowej ramki
\usepackage[pdfborder={0 0 0 0}]{hyperref}
% - dołączanie obrazków
\usepackage{graphics}
% - szersza strona
\usepackage[nofoot,hdivide={2cm,*,2cm},vdivide={2cm,*,2cm}]{geometry}
\usepackage{lmodern}
\frenchspacing
% - brak numerów stron
\pagestyle{empty}

% dane autora
\author{Julia Majkowska}
\title{Krótki opis projektu}
\date{\today}

% początek dokumentu
\begin{document}
 \maketitle
 Program umożliwiający zamianę mapy wysokości na kolorową mapę hipsometryczną. \\
 Będzie umożliwiał :
 
 - wybór w nowym okienku pliku źródłowego tablicy wysokosci(obslugiwane będą formaty DTED( Level 0,1,2), SRTM(pliki binarne HGT, TIFF, PNG)
 
 - stworzenie własnego zestawu poziomic i odpowiadającej mu palety, przez dodawanie i usuwanie poziomic (za pomocą odpowiednych przycisków i przypisywanie im wartosci w metrach ( wpisując ją do pola tekstowgo -GtkEntry) oraz koloru (przez kliknięcie odpowiedniego przycisku i wybranie koloru myszką w okienku -skorzystam ze struktury GtkColorChooserButton)
 
 - zapis poziomic i palety do istniejącego lub nowego pliku,wybranego lub wpisanego w odpowiednim okienku (skorzystam ze struktury GtkFileChooserDialog)
 
 - wczytanie zestawu poziomic i palety , zastępując stworzone w oknie poziomice, z pliku wybranego w nowym okienku
 
 - wczytanie poziomic dodając je do palety w okienku, z pliku wybranego w nowym okienku
 
 - skorzystania z domyślnej palety kolorów ( po kliknięciu odpowiedniego przycisku aktulana paleta zostanie zastąpiona domyslną)
 
 - podgląd mapy dla aktualnej palety i poziomic w nowym oknie po kliknięciu odpowiedniego przycisku (struktura z tablicą wysokości będzie konwertowana do struktury GtkPixbuf i wyświetlana z pomocą funkcji gtk_image_new_from_pixbuf).
 
 - zapis mapy do pliku png, przez kliknięcie przycisku i wybranie lub wpisanie pliku w nowym okienku 
 

Wszelkie pliki (palety i mapy)  dla programu musza znajdować się w tym samym katalogu co program. Nowo utworzone pliki również będą się tworzyć w tym katalogu co program.

 

\end{document}

